\chapter{RESULTADOS}

\lipsum[1-1]\autocite{texbook,latex:companion}.

En la \autoref{fig:fig2} podemos observar un ejemplo de figura.

\begin{figure}[H]
	\centering
	\includegraphics[width=\textwidth,height=0.3\textwidth]{overleaf}
	\caption{Mi figura}\label{fig:fig2}
\end{figure}

En la \autoref{eq:ecuacion1} podemos observar un ejemplo de ecuación.

\begin{eq}[H]
	\caption{Mi ecuación}\label{eq:ecuacion1}
	\[
		\mathcal L_{\mathcal T}(\vec{\lambda})
		= \sum_{(\mathbf{x},\mathbf{s})\in \mathcal T}
		\log P(\mathbf{s}\mid\mathbf{x}) - \sum_{i=1}^m
		\frac{\lambda_i^2}{2\sigma^2}
	\]
\end{eq}

% Ejemplo de una Figura en otra página se puede ver en la \vref{fig:fig3}

% \begin{figure}[H]
% 	\centering
% 	\includegraphics[width=\textwidth,height=0.5\textwidth]{example-image}
% 	\caption{Figura en otra página}\label{fig:fig3}
% \end{figure}

Este es un ejemplo de matemática dentro de línea \(x^2 + y^2 = z^2\).

% Este es el ejemplo de una cita múltiple \autocite{texbook,latex:companion}.